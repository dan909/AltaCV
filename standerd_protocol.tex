
\documentclass[10pt,a4paper]{protocol}

% Change the page layout if you need to
\geometry{left=1cm,right=9cm,marginparwidth=6.8cm,marginparsep=1.2cm,top=1cm,bottom=1cm}

% Change the font if you want to.

% If using pdflatex:
\usepackage[utf8]{inputenc}
\usepackage[T1]{fontenc}
\usepackage[default]{lato}

% If using xelatex or lualatex:
% \setmainfont{Lato}

% Change the colours if you want to
\definecolor{Aber}{HTML}{58318f}
\definecolor{IBERSgreen}{HTML}{319866}
\definecolor{Black}{HTML}{111111}
\definecolor{LightGrey}{HTML}{515c50}
\colorlet{heading}{IBERSgreen}
\colorlet{accent}{Aber}
\colorlet{emphasis}{Black}
\colorlet{body}{LightGrey}

% Change the bullets for itemize and rating marker
% for \risk if you want to
\renewcommand{\itemmarker}{{\small\textbullet}}
\renewcommand{\ratingmarker}{\faSpinner}

%% sample.bib contains your publications
\addbibresource{sample.bib}

\begin{document}
\name{A Standerd Protocol}
\tagline{This is the simple version of a complex protocol}
\made{October 2 2017}
\logo{6.5cm}{"Logo"}


\docinfo{%
  % can add more \addedtopeople
  \madeby{Danny Awty-Carroll}{dga1@aber.ac.uk}{October 2, 2017}
  \addedto{Danny Awty-Carroll}{dga1@aber.ac.uk}{October 3, 2017}
}


\purpose{
	Lorem ipsum dolor sit amet, consectetur adipiscing elit. Nunc porta dui a fermentum varius. Aliquam cursus urna sit amet urna sollicitudin, vel pellentesque turpis imperdiet. Aenean lectus neque, rhoncus vel.
} % add a short discription of the purpose for this protocol


%% Make the header extend all the way to the right, if you want.
\begin{fullwidth}
\makeheader
\end{fullwidth}

%% Provide the file name containing the sidebar contents as an optional parameter to \need.
%% You can always just use \marginpar{...} if you do
%% not need to align the top of the contents to any
%% \need title in the "main" bar.
\need[Materials]{Protocol}

\step{1}{Lorem ipsum dolor sit amet, consectetur}{20}
\begin{itemize}
	\item Take one of x and then
	\item Then do the next thing
	\item Then do the next thing
	\item Then do the next thing
\end{itemize}
\divider

\step{2}{Pellentesque habitant morbi tristique senectus}{35}
\begin{itemize}
	\item Then do the next thing
	\item Then do the next thing
	\item Then do the next thing
\end{itemize}
\divider

\step{3}{habitant morbi tristique senectus}{40}
\begin{itemize}
	\item Then do the next thing
	\item Then do the next thing
\end{itemize}
\divider

\step{4}{Ut quis orci luctus, efficitur sem vitae}{25}
\begin{itemize}
	\item Do this using the method in the bib file (\cite{einstein})
	\item Then do the next thing
\end{itemize}
\divider

\step{5}{orbi tristique senectus}{30}
\begin{itemize}
	\item Then do the next thing
	\item Then do the next thing
	\item Then do the next thing
\end{itemize}
\divider

\step{6}{ipsum dolor sit amet}{15}
\begin{itemize}
	\item Then do the next thing as done by \cite{einstein}
	\item Then do the next thing
	\item Then do the next thing
	\item Then do the next thing
\end{itemize}
\divider


\clearpage

\need[otherinfo]{Sources}

\nocite{*}

\printbibliography


\divider


%% If the NEXT page doesn't start with a \need but you'd
%% still like to add a sidebar, then use this command on THIS
%% page to add it. The optional argument lets you pull up the
%% sidebar a bit so that it looks aligned with the top of the
%% main column.
% \addnextpagesidebar[-1ex]{page3sidebar}


\end{document}
